\documentclass{article}

\usepackage[french]{babel}
\usepackage[T1]{fontenc}
\usepackage[utf8]{inputenc}
\usepackage{amssymb}
\usepackage{amsmath}
\usepackage{array}
\usepackage{xcolor}
\usepackage{listings}

\newcommand{\plagiatWikipedia}{\footnote{\underline{Source} : Wikipedia}}
\newcommand{\plagiatBibmath}{\footnote{\underline{Source} : Bibmath}}

\lstset{
    numbers=left
}

\title{Complexité}
\author{235711131723}

\begin{document}
    \maketitle

    \section{Différents algorithmes de tri}

    \subsection{Tri à bulle}

    \paragraph{Principe.}

    \begin{itemize}
        \item On parcourt le tableau et on permute une case avec la suivante si elles ne sont pas dans le bon ordre
        \item À la fin du parcours, si on fait une permutation, on recommence le parcours.
    \end{itemize}

    \paragraph{Théorème.} Le tri à bulle termine correctement en $\Theta(n^2)$.

    \paragraph{Algorithme.}

    \underline{Entrée} : Tableau d'entiers T de taille $n$.

    \begin{lstlisting}
Bool Permutation = Vrai
Tant Que(Permutation):
    Permutation = Faux
    Pour(i = 0; i < n - 1; i = i + 1):
        Si(T[i] > T[i + 1]):
            temp = T[i]
            T[i] = T[i + 1]
            T[i + 1] = temp
            Permutation = Vrai
        Fin Si
    Fin Pour
Fin Tant Que
    \end{lstlisting}

    \paragraph{Démonstration du théorème.}

    Posons l'invariant ($j \in \mathbb{N}$) $H(i)$ : "Après $i$ exécution de la boucle \textbf{Tant Que}, les $j$ dernières cases du tableau contiennent les plus grands éléments triés par ordre croissant."\\

    \begin{itemize}
        \item \textbf{Initialisation de l'invariant.} Quand on arrive à la première exécution de la ligne 3, on a $j = 0$ et l'invariant est totalement vrai.
    \end{itemize}
\end{document}