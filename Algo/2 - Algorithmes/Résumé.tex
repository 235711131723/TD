\documentclass{article}

\usepackage[french]{babel}
\usepackage[T1]{fontenc}
\usepackage[utf8]{inputenc}
\usepackage{amssymb}
\usepackage{amsmath}
\usepackage{array}
\usepackage{xcolor}
\usepackage{listings}
\usepackage{verbatim}

\newcommand{\plagiatWikipedia}{\footnote{\underline{Source} : Wikipedia}}
\newcommand{\plagiatBibmath}{\footnote{\underline{Source} : Bibmath}}

\lstset{
    numbers=left
}

\title{Complexité}
\author{235711131723}

\begin{document}
    \maketitle

    \section{Différents algorithmes de tri}

    \subsection{Tri à bulle}

    \paragraph{Principe.}

    \begin{itemize}
        \item On parcourt le tableau et on permute une case avec la suivante si elles ne sont pas dans le bon ordre
        \item À la fin du parcours, si on fait une permutation, on recommence le parcours.
    \end{itemize}

    \paragraph{Théorème.} Le tri à bulle termine correctement en $\Theta(n^2)$.

    \paragraph{Algorithme.}

    \underline{Entrée} : Tableau d'entiers T de taille $n$.

    \begin{lstlisting}
Entier temp
Bool Permutation := Vrai
Tant Que(Permutation):
    Permutation := Faux
    Pour(i := 0; i < n - 1; i := i + 1):
        Si(T[i] > T[i + 1]):
            temp := T[i]
            T[i] := T[i + 1]
            T[i + 1] := temp
            Permutation := Vrai
        Fin Si
    Fin Pour
Fin Tant Que
    \end{lstlisting}

    \paragraph{Démonstration du théorème.}

    Posons $j \in \mathbb{N}$, pour $0 \le k \le n - 1$, $H(k)$ : "Après $k$ exécution de la boucle \textbf{Tant Que}, les $j$ dernières cases du tableau contiennent les $j$ plus grands éléments triés par ordre croissant.".\\

    \begin{itemize}
        \item \textbf{Initialisation de l'invariant.} Quand on arrive à la première exécution de la ligne 3, on a $j = 0$ et l'invariant est trivialement vrai.
        \item \textbf{Sortie de la boucle \underline{Tant Que}.} On a \verb|Permutation := Faux|. Puisqu'on avait initialement \verb|Permutation := Vrai| alors $j > 0$.
    \end{itemize}

    \paragraph{Tableau déjà entièrement trié.} Supposons que $j = n$. Alors l'invariant $H(k)$ devient, pour $0 \le k \le n - 1$, $H(k)$ : "Le tableau est trié par ordre croissant à la k-ième itération.".

    \begin{itemize}
        \item Lors de la première itération de la boucle \textbf{Tant Que}, le test de la ligne 6 est toujours faux.
        \item On sort de cette boucle à la fin de la première itération. Je termine en $O(n)$.
        \item Je termine.
    \end{itemize}

    \paragraph{Invariant de la boucle \textbf{Pour}.} Supposons l'invariant $H(k)$ vrai pour $k \in \mathbb{N}$ et supposons que nous sommes à la ligne 4 du code avec \verb|Permutation := Vrai|. Montrons que $H(k+1)$ est vrai.\\

    Posons $M = \max\limits_{i \in [\![0, n - (j + 1)]\!]} T[i]$ et $m$ tel que $T[m] = M$. Posons $H'(i)$ : "Les dernières cases du tableau sont triés par ordre croissant et $T[\min(n - (j + 1), \max(i, m))] = M$.".

    \paragraph{Initialisation.} Pour $i = 0$ et $H(k)$, les $j$ dernières cases sont triées et contiennent $T[\min(n - (j + 1), \max(i, m))] = M$. De plus, $\min(n - (j + 1), \max(i, m)) = m$ et $T[m] = M$.

    \subsection{Tri par insertion}

    \paragraph{Principe.}
\end{document}